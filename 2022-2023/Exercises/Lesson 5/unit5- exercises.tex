\documentclass[]{article}

\usepackage{enumitem}
\usepackage{geometry}[top = 1cm, bottom = 2cm, left = 1.5cm, right = 1.5cm]
\usepackage{listings}

%opening
\title{Development 8 - Exercises\\Unit 5}
\author{}
\date{}

\newcounter{ExerciseCount}
\setcounter{ExerciseCount}{1}

\newcommand{\functionEx}[3]{
  Implement a function\\\\
   \texttt{let #1 = #2}\\\\ #3
}

\newcommand{\exercise}[1]{\noindent \textbf{Exercise \theExerciseCount:}\\\\ #1 \addtocounter{ExerciseCount}{1}
}

\lstset{
  breaklines = true,
  basicstyle = \ttfamily,
  tabsize = 2
}

\begin{document}
\maketitle
\noindent
For these exercises use the following type definitions:\\\\

\noindent
\textbf{BINARY TREE:}
\begin{lstlisting}
type BinaryTreeData<a> = {
  kind: "empty"
} | {
  kind: "node",
  value: a
  left: BinaryTree<a>,
  right: BinaryTree<a>
}  
\end{lstlisting}

\noindent
\textbf{FUNCTOR:}
\begin{lstlisting}
interface Functor<F, G, a, b> {
  map: (this: F, f: (x: a) => b) => G
}
\end{lstlisting}

\noindent
Moreover, try to express the operations on the data structures as immutable methods by extending the basic data structure with a record of functions.\\


\exercise{
  \functionEx{tryFind}{<a>(value: a) => (tree: BinaryTree<a>): Option<BinaryTree<a>>}{
    that looks up for an element in a binary search tree. If the element is not found the function returns \texttt{None}.
}}\\

\exercise{
  \functionEx{insert}{<a>(value: a) => (tree: BinaryTree<a>): BinaryTree<a>}{
    that inserts a new element in a binary search tree.
}}\\

\exercise{
  \functionEx{inorderFold}{<a, state>(f: (s: state) => (x: a) => state) \\=> (init: state) => (tree: BinaryTree<a>): state}{
    that carries an accumulator through the in-order traversal of the binary search tree and updates its value by executing the function \texttt{f}.
}}\\

\exercise{
  \functionEx{treeMap}{<a, b>(f: (x: a) => b) => (tree: BinaryTree<a>): BinaryTree<b>}{
    that applies the function \texttt{f} to each node in a binary search tree by performing the in-order traversal. The function outputs a tree containing the results of the function application. SUGGESTION: to preserve the binary search property use a fold on the tree to accumulate the result.
}}\\

\exercise{Extend the \texttt{Option} data type to be a functor using the function record \texttt{Functor<F, G, a, b>}}\\

\exercise{Extend the \texttt{List} data type to be a functor using the function record \texttt{Functor<F, G, a, b>}}\\

\exercise{Extend the \texttt{Tree} data type to be a functor using the function record \texttt{Functor<F, G, a, b>}}\\



\end{document}